% !TeX program = xelatex   % 告诉 VS Code/编辑器用 xelatex 编译,支持中文

\documentclass[UTF8]{ctexart} 
% \documentclass:声明“文档类型”,最常用的是 article(文章)
% ctexart:是 article 的中文版本,支持中文输入
% [UTF8]:表示使用 UTF-8 编码(一般文件默认就是 UTF-8)

% ---------- 常用宏包 ----------
\usepackage{geometry}   % 控制页面大小和边距
\usepackage{graphicx}   % 插入图片(\includegraphics)
\usepackage{amsmath}    % 数学公式(如 \frac a b 分数)
\usepackage{listings}   % 插入代码环境
\usepackage{xcolor}     % 定义颜色
\usepackage{hyperref}   % 插入超链接
\usepackage{fancyhdr}   % 美化页眉页脚

% ---------- 页面设置 ----------
\geometry{a4paper, scale=0.8}     
% a4paper:A4 纸张
% scale=0.8:把版心缩小到 80%,让边距更宽,看起来舒服

\pagestyle{fancy}                 
% 设置页眉页脚风格为 fancyhdr 提供的“fancy”模式
\fancyhf{}                        
% 清空默认的页眉页脚
\fancyhead[C]{系统开发工具基础 实验报告} 
% 页眉(head),C 表示居中,内容是标题
\fancyfoot[C]{\thepage}           
% 页脚(foot),C 表示居中,内容是页码 \thepage

% ---------- 报告信息 ----------
\title{《系统开发工具基础》实验报告\\ }  %此处填写报告标题 
% \title{} 定义标题,\\ 表示换行
\author{姓名:马一诺 \\ 学号:24020007088 \\ 专业:计算机科学与技术}
% \author{} 作者信息,\quad 表示一个小空格
\date{\today} 
% \date{} 定义日期,这里用 \today 自动生成当天日期

% ==================================================
\begin{document}
% 文档开始

\maketitle 
% 生成封面标题,使用上面 \title, \author, \date 定义的信息

\tableofcontents 
% 自动生成目录(根据 \section, \subsection 生成)
\newpage
% 换页

\section{实验目的}
% \section{}:一级标题
简要说明实验的目标,比如:掌握 Git 基本使用、掌握 LaTeX 文档编写。

\section{实验环境}
- 硬件:xxx  
- 软件:Windows 11, VS Code, TeX Live 2024  
% “-” 自动生成列表,两个空格表示换行

\section{实验原理}
解释实验相关的理论知识或工具原理,例如 Git 的版本控制思想。

\section{实验步骤}
1. 安装 Git  
2. 初始化仓库  
3. 编写 LaTeX 文件  
4. 推送到 GitHub  
% 用 “1. 2. 3.” 自动生成有序列表

\section{实验结果}
这里可以插入实验截图:  
\begin{figure}[h]
  \centering
  \includegraphics[width=0.6\textwidth]{example-image}
  \caption{GitHub 成功推送页面截图}
\end{figure}
% \begin{figure} 环境插入图片
% [h] 表示 here(尽量放在当前位置)
% \centering:居中
% \includegraphics[width=0.6\textwidth]{文件名}:插入图片,宽度是正文宽度的 60%
% \caption{}:图片标题

\section{实验总结}
写一些收获、体会和遇到的问题。

\section{参考文献}
\begin{thebibliography}{99}
\bibitem{git} Git 官方文档: \url{https://git-scm.com/doc}
\bibitem{latex} LaTeX Wikibook: \url{https://en.wikibooks.org/wiki/LaTeX}
\end{thebibliography}
% thebibliography 环境:生成参考文献
% {99} 表示最多两位数的编号
% \bibitem{标识符} 文献条目
% \url{} 插入超链接

\end{document}
% 文档结束
